\usepackage{graphicx,graphics,epsfig,subfigure,times,bm,bbm,amssymb,amsmath,amsfonts,amsthm,amscd,mathrsfs,MnSymbol,accents}
\usepackage[matrix,frame,arrow]{xypic}
\usepackage[pdftex]{color}

\usepackage{braket}%Dirac Notation in QM
\usepackage{enumerate}
%    Q-circuit version 1.2
%    Copyright (C) 2004  Steve Flammia & Bryan Eastin, 4/23/06
%    This program is free software; you can redistribute it and/or modify
%    it under the terms of the GNU General Public License as published by
%    the Free Software Foundation; either version 2 of the License, or
%    (at your option) any later version.
%
%    This program is distributed in the hope that it will be useful,
%    but WITHOUT ANY WARRANTY; without even the implied warranty of
%    MERCHANTABILITY or FITNESS FOR A PARTICULAR PURPOSE.  See the
%    GNU General Public License for more details.
%
%    You should have received a copy of the GNU General Public License
%    along with this program; if not, write to the Free Software
%    Foundation, Inc., 59 Temple Place, Suite 330, Boston, MA  02111-1307  USA

\usepackage[matrix,frame,arrow]{xy}
\usepackage{amsmath}
\newcommand{\bra}[1]{\left\langle{#1}\right\vert}
\newcommand{\ket}[1]{\left\vert{#1}\right\rangle}
    % Defines Dirac notation.
\newcommand{\qw}[1][-1]{\ar @{-} [0,#1]}
    % Defines a wire that connects horizontally.  By default it connects to the object on the left of the current object.
    % WARNING: Wire commands must appear after the gate in any given entry.
\newcommand{\qwx}[1][-1]{\ar @{-} [#1,0]}
    % Defines a wire that connects vertically.  By default it connects to the object above the current object.
    % WARNING: Wire commands must appear after the gate in any given entry.
\newcommand{\cw}[1][-1]{\ar @{=} [0,#1]}
    % Defines a classical wire that connects horizontally.  By default it connects to the object on the left of the current object.
    % WARNING: Wire commands must appear after the gate in any given entry.
\newcommand{\cwx}[1][-1]{\ar @{=} [#1,0]}
    % Defines a classical wire that connects vertically.  By default it connects to the object above the current object.
    % WARNING: Wire commands must appear after the gate in any given entry.
\newcommand{\gate}[1]{*{\xy *+<.6em>{#1};p\save+LU;+RU **\dir{-}\restore\save+RU;+RD **\dir{-}\restore\save+RD;+LD **\dir{-}\restore\POS+LD;+LU **\dir{-}\endxy} \qw}
    % Boxes the argument, making a gate.
\newcommand{\meter}{\gate{\xy *!<0em,1.1em>h\cir<1.1em>{ur_dr},!U-<0em,.4em>;p+<.5em,.9em> **h\dir{-} \POS <-.6em,.4em> *{},<.6em,-.4em> *{} \endxy}}
    % Inserts a measurement meter.
\newcommand{\measure}[1]{*+[F-:<.9em>]{#1} \qw}
    % Inserts a measurement bubble with user defined text.
\newcommand{\measuretab}[1]{*{\xy *+<.6em>{#1};p\save+LU;+RU **\dir{-}\restore\save+RU;+RD **\dir{-}\restore\save+RD;+LD **\dir{-}\restore\save+LD;+LC-<.5em,0em> **\dir{-} \restore\POS+LU;+LC-<.5em,0em> **\dir{-} \endxy} \qw}
    % Inserts a measurement tab with user defined text.
\newcommand{\measureD}[1]{*{\xy*+=+<.5em>{\vphantom{\rule{0em}{.1em}#1}}*\cir{r_l};p\save*!R{#1} \restore\save+UC;+UC-<.5em,0em>*!R{\hphantom{#1}}+L **\dir{-} \restore\save+DC;+DC-<.5em,0em>*!R{\hphantom{#1}}+L **\dir{-} \restore\POS+UC-<.5em,0em>*!R{\hphantom{#1}}+L;+DC-<.5em,0em>*!R{\hphantom{#1}}+L **\dir{-} \endxy} \qw}
    % Inserts a D-shaped measurement gate with user defined text.
\newcommand{\multimeasure}[2]{*+<1em,.9em>{\hphantom{#2}} \qw \POS[0,0].[#1,0];p !C *{#2},p \drop\frm<.9em>{-}}
    % Draws a multiple qubit measurement bubble starting at the current position and spanning #1 additional gates below.
    % #2 gives the label for the gate.
    % You must use an argument of the same width as #2 in \ghost for the wires to connect properly on the lower lines.
\newcommand{\multimeasureD}[2]{*+<1em,.9em>{\hphantom{#2}}\save[0,0].[#1,0];p\save !C *{#2},p+LU+<0em,0em>;+RU+<-.8em,0em> **\dir{-}\restore\save +LD;+LU **\dir{-}\restore\save +LD;+RD-<.8em,0em> **\dir{-} \restore\save +RD+<0em,.8em>;+RU-<0em,.8em> **\dir{-} \restore \POS !UR*!UR{\cir<.9em>{r_d}};!DR*!DR{\cir<.9em>{d_l}}\restore \qw}
    % Draws a multiple qubit D-shaped measurement gate starting at the current position and spanning #1 additional gates below.
    % #2 gives the label for the gate.
    % You must use an argument of the same width as #2 in \ghost for the wires to connect properly on the lower lines.
\newcommand{\control}{*!<0em,.025em>-=-{\bullet}}
    % Inserts an unconnected control.
\newcommand{\controlo}{*-<.21em,.21em>{\xy *=<.59em>!<0em,-.02em>[o][F]{}\POS!C\endxy}}
    % Inserts a unconnected control-on-0.
\newcommand{\ctrl}[1]{\control \qwx[#1] \qw}
    % Inserts a control and connects it to the object #1 wires below.
\newcommand{\ctrlo}[1]{\controlo \qwx[#1] \qw}
    % Inserts a control-on-0 and connects it to the object #1 wires below.
\newcommand{\targ}{*!<0em,.019em>=<.79em,.68em>{\xy {<0em,0em>*{} \ar @{ - } +<.4em,0em> \ar @{ - } -<.4em,0em> \ar @{ - } +<0em,.36em> \ar @{ - } -<0em,.36em>},<0em,-.019em>*+<.8em>\frm{o}\endxy} \qw}
    % Inserts a CNOT target.
\newcommand{\qswap}{*=<0em>{\times} \qw}
    % Inserts half a swap gate. 
    % Must be connected to the other swap with \qwx.
\newcommand{\multigate}[2]{*+<1em,.9em>{\hphantom{#2}} \qw \POS[0,0].[#1,0];p !C *{#2},p \save+LU;+RU **\dir{-}\restore\save+RU;+RD **\dir{-}\restore\save+RD;+LD **\dir{-}\restore\save+LD;+LU **\dir{-}\restore}
    % Draws a multiple qubit gate starting at the current position and spanning #1 additional gates below.
    % #2 gives the label for the gate.
    % You must use an argument of the same width as #2 in \ghost for the wires to connect properly on the lower lines.
\newcommand{\ghost}[1]{*+<1em,.9em>{\hphantom{#1}} \qw}
    % Leaves space for \multigate on wires other than the one on which \multigate appears.  Without this command wires will cross your gate.
    % #1 should match the second argument in the corresponding \multigate. 
\newcommand{\push}[1]{*{#1}}
    % Inserts #1, overriding the default that causes entries to have zero size.  This command takes the place of a gate.
    % Like a gate, it must precede any wire commands.
    % \push is useful for forcing columns apart.
    % NOTE: It might be useful to know that a gate is about 1.3 times the height of its contents.  I.e. \gate{M} is 1.3em tall.
    % WARNING: \push must appear before any wire commands and may not appear in an entry with a gate or label.
\newcommand{\gategroup}[6]{\POS"#1,#2"."#3,#2"."#1,#4"."#3,#4"!C*+<#5>\frm{#6}}
    % Constructs a box or bracket enclosing the square block spanning rows #1-#3 and columns=#2-#4.
    % The block is given a margin #5/2, so #5 should be a valid length.
    % #6 can take the following arguments -- or . or _\} or ^\} or \{ or \} or _) or ^) or ( or ) where the first two options yield dashed and
    % dotted boxes respectively, and the last eight options yield bottom, top, left, and right braces of the curly or normal variety.
    % \gategroup can appear at the end of any gate entry, but it's good form to pick one of the corner gates.
    % BUG: \gategroup uses the four corner gates to determine the size of the bounding box.  Other gates may stick out of that box.  See \prop. 
\newcommand{\rstick}[1]{*!L!<-.5em,0em>=<0em>{#1}}
    % Centers the left side of #1 in the cell.  Intended for lining up wire labels.  Note that non-gates have default size zero.
\newcommand{\lstick}[1]{*!R!<.5em,0em>=<0em>{#1}}
    % Centers the right side of #1 in the cell.  Intended for lining up wire labels.  Note that non-gates have default size zero.
\newcommand{\ustick}[1]{*!D!<0em,-.5em>=<0em>{#1}}
    % Centers the bottom of #1 in the cell.  Intended for lining up wire labels.  Note that non-gates have default size zero.
\newcommand{\dstick}[1]{*!U!<0em,.5em>=<0em>{#1}}
    % Centers the top of #1 in the cell.  Intended for lining up wire labels.  Note that non-gates have default size zero.
\newcommand{\Qcircuit}[1][0em]{\xymatrix @*[o] @*=<#1>}
    % Defines \Qcircuit as an \xymatrix with entries of default size 0em.  The optional argument, #1, is for use with clusters, and allows you
    % to fix the size of the nodes.  I would not advise using it with normal circuits.
\newcommand{\node}[2][]{{\begin{array}{c} \ _{#1}\  \\ {#2} \\ \ \end{array}}\drop\frm{o} }
    % When Qcircuit has been passed the optional argument for cluster states, this command produces a round node of the size specified in that
    % argument.  The optional argument #2 specifies the contents of a node, while optional argument #1 is a secondary label.  
\newcommand{\link}[2]{\ar @{-} [#1,#2]}
    % Draws a wire or connecting line to the element #1 rows down and #2 columns forward.
\newcommand{\pureghost}[1]{*+<1em,.9em>{\hphantom{#1}}}
    % Same as \ghost except it omits the wire leading to the left.  %Quantum Circuits
\usepackage{multirow}
\usepackage{textcomp}
\usepackage{booktabs}
\usepackage[english]{babel}
\usepackage{url}
\usepackage{appendix}
\usepackage[counter-within=section,counter-format=ch.se.qu]{exsheets}
\usepackage{makeidx}
\usepackage[pdfstartview=FitH]{hyperref}
\hypersetup{
    colorlinks=true,       % false: boxed links; true: colored links
    linkcolor=cyan,          % color of internal links
    citecolor=magenta,        % color of links to bibliography
    filecolor=magenta,      % color of file links
    urlcolor=cyan,           % color of external links
    runcolor=cyan
}
\usepackage[capitalise]{cleveref}
\usepackage{cancel}
\usepackage{mathtools}

\renewcommand*{\theHsection}{\thesection}
\renewcommand*{\theHsubsection}{\thesubsection}
\usepackage{hypernat}
\usepackage{environ}
\usepackage{ragged2e}
\usepackage{caption}



% Define a new environment for tables with customized captions
\newenvironment{mytable}
  {% Begin code
   \begin{table}[ht]
   \centering
  }
  {% End code
   \end{table}
  }

%% Override the caption formatting of cup6a
\makeatletter
\long\def\make@table@caption#1#2{\vskip 10\p@%
    \setbox\@tempboxa\hbox{{\rm #1.\hspace{0.5em}\itshape #2}}%
    \ifdim \wd\@tempboxa >\hsize
    { \rm #1.\hspace{0.5em}{\itshape #2}\par}%
    \else
    \ifSFB@indentsty
    \hbox to\hsize{\box\@tempboxa\hfill}\par
    \else
    \hbox to\hsize{\hfil\box\@tempboxa\hfil}\par
    \fi
    \fi
    \vspace*{2.5\p@}\par
}
\makeatother

% Custom command to create a caption with specific formatting
\newcommand{\mycaption}[1]{%
  \caption{%
     #1
}%
}

\makeindex

\usepackage{paralist} 
\usepackage{psfrag} 
\usepackage{cupbookpatch11}
\usepackage{color}
\usepackage{fancyhdr}
\usepackage[PetersLenny]{fncychap}
\usepackage{tikz}
\usetikzlibrary{positioning}
\def\checkmark{\tikz\fill[scale=0.4](0,.35) -- (.25,0) -- (1,.7) -- (.25,.15) -- cycle;}

\setlength{\unitlength}{1cm}

\setlength{\marginparwidth}{1.5in}
\setlength{\textheight}{9.5in}
\setlength{\textwidth}{6in}
\setlength{\topmargin}{-.5in}
\setlength{\oddsidemargin}{.16in}
\setlength{\evensidemargin}{.2in} 
\setlength{\headsep}{.5in} 
\setlength{\parindent}{1.5pc}

\renewcommand{\baselinestretch}{1.02}
\newcommand{\HRule}{\hskip -.6cm \rule{\linewidth}{0.2mm}}

%superscript footnote
\makeatletter
\def\@makefnmark{\hbox{\@textsuperscript{\normalfont\@thefnmark}}}
\makeatother

% for algorithms
\usepackage{algorithm}
\usepackage{algpseudocode}

\newtheorem{thm}{Theorem}[section]
\newtheorem{mydef}{Definition}[section]
\newtheorem{col}{Corollary}[section]
\newtheorem{pos}{Postulate}[section]

\newtheorem{mytheorem}{Theorem}[section]
\newtheorem{mylemma}{Lemma}[section]
\newtheorem{mycorollary}{Corollary}[section]
\newtheorem{myproposition}{Proposition}[section]
\newtheorem{myclaim}{Claim}[section]
\newtheorem{mydefinition}{Definition}[section]
\newtheorem{myassumption}{Assumption}[section]
\newtheorem{mypostulate}{Postulate}[]

% Primed postulates
\makeatletter
\newcommand{\neutralize}[1]{\expandafter\let\csname c@#1\endcsname\count@}
\makeatother

\newenvironment{mypostulateDM}[1]
{\renewcommand{\thepos}{\ref*{#1}$'$}%
    \neutralize{ps}\phantomsection%
    \begin{pos}}%
    {\end{pos}}%

\newenvironment{mypostulateDM2}[1]
{\renewcommand{\thepos}{\ref*{#1}$''$}%
    \neutralize{ps}\phantomsection%
    \begin{pos}}%
    {\end{pos}}%

\newcommand{\msf}{\mathsf}
\newcommand{\mrm}{\mathrm}
\newcommand{\mbf}{\mathbf}
\newcommand{\mbb}{\mathbb}
\newcommand{\mc}{\mathcal}
\newcommand{\tx}[1]{\text{#1}}
\newcommand{\derv}[3]{\frac{d^{#3}#1}{d#2^{#3}}}					% DERIVATIVE
\newcommand{\pdv}[2]{\frac{\partial#1}{\partial#2}}
\newcommand{\half}{\tfrac{1}{2}}

\newcommand{\bes} {\begin{subequations}}
\newcommand{\ees} {\end{subequations}}
\newcommand{\ba}{\begin{eqnarray}}
\newcommand{\ea}{\end{eqnarray}}
\newcommand{\bea} {\begin{eqnarray}}
\newcommand{\eea} {\end{eqnarray}}
\newcommand{\beq}{\begin{equation}}
\newcommand{\eeq}{\end{equation}}


\newcommand{\red}[1]{\textcolor{red}{#1}} 
\newcommand{\green}[1]{\textcolor{black}{#1}}
\newcommand{\blue}[1]{\textcolor{blue}{#1}}

\newcommand{\expv}[1]{\langle #1\rangle}							% EXPECTATION VALUE
\newcommand{\ph}{\ensuremath{\varphi}}
\newcommand{\eps}{\ensuremath{\varepsilon}} %"nice" epsilon
\newcommand{\prima}[1]{\ensuremath{#1^\prime}} %add a prime to something
\newcommand{\R}{\ensuremath{\mathbb{R}}} %real numbers
\newcommand{\C}{\ensuremath{\mathbb{C}}} %complex numbers
\newcommand{\Q}{\ensuremath{\mathbb{Q}}} %rational numbers
\newcommand{\Z}{\ensuremath{\mathbb{Z}}} %integers
\newcommand{\N}{\ensuremath{\mathbb{N}}} %natural numbers
\newcommand{\e}{\ensuremath{{e}}} %e:=lim_{n\to\infty}(1+1/n)^n
\newcommand{\ii}{\ensuremath{{i}}}%i:=\sqrt{-1}
\newcommand{\abs}[1]{\ensuremath{\left|#1\right|}} %absolute value
\newcommand{\norm}[1]{\ensuremath{\left\|#1\right\|}} %norm
\newcommand{\opU}{\ensuremath{{{U}}}}
\newcommand{\opH}{\ensuremath{{{H}}}}
\newcommand{\dfsh}{\mathrm{DFS}}
\newcommand{\mcal}[1]{\mathcal{#1}}
\newcommand{\ma}[1]{\mathcal{#1}}
\newcommand{\expect}[3]{\<{#1}|{#2}|{#3}\>}
\newcommand{\ave}[1]{\<{#1}\>}
\newcommand{\sinc}{\mathrm{sinc}}
\newcommand{\comm}[2]{\left[ #1, #2 \right]}
\newcommand{\acomm}[2]{\left\{ #1, #2 \right\}}
\newcommand{\grad}{\vec{\nabla}}
\newcommand{\bqty}[1]{\left( #1 \right)}
\newcommand{\Bqty}[1]{\left\{ #1 \right\}}
\newcommand{\pqty}[1]{\left( #1 \right)} % Define the \pqty command
\newcommand{\Pqty}[1]{\left[ #1 \right]} % Define the \Pqty command
\newcommand{\mel}[3]{\langle #1 | #2 | #3 \rangle}
\newcommand{\dyad}[2]{
  \if\relax\detokenize{#2}\relax
    \ket{#1}\bra{#1} % Treat single input as double input
  \else
    \ket{#1}\bra{#2} % Treat double inputs normally
  \fi
} % Define the modified \dyad command

\renewcommand{\Re}{\mathrm{Re}}
\renewcommand{\Im}{\mathrm{Im}}

\newcommand{\bp}{\bar{\psi}}

\newcommand{\pen}[1]{\left(#1\right)}								% PARENTHESIS
\newcommand{\ben}[1]{\left[#1\right]}								% BRACKETS
\newcommand{\cen}[1]{\left\{#1\right\}}								% CURLY BRACKETS

\newcommand{\ignore}[1]{}
\newcommand{\ignoreforclass}[1]{}
\newcommand{\mcHS}{\mathcal{H}_S}
\newcommand{\mcHB}{\mathcal{H}_B}
\newcommand{\mcHR}{\mathcal{H}_R}
\newcommand{\mA}{\mathcal{A}}
\newcommand{\mcB}{\mathcal{B}}
\newcommand{\mB}{\mathcal{B}}
\newcommand{\mC}{\mathcal{C}}
\newcommand{\mG}{\mathcal{G}}
\newcommand{\mcH}{\mathcal{H}}
\newcommand{\mI}{\mathcal{I}}
\newcommand{\mJ}{\mathcal{J}}
\newcommand{\mK}{\mathcal{K}}
\newcommand{\mL}{\mathcal{L}}
\newcommand{\mN}{\mathcal{N}}
\newcommand{\mO}{\mathcal{O}}
\newcommand{\mP}{\mathcal{P}}
\newcommand{\mQ}{\mathcal{Q}}
\newcommand{\mU}{\mathcal{U}}
\newcommand{\mV}{\mathcal{V}}
\newcommand{\mZ}{\mathcal{Z}}

\newcommand{\cU}{\mathcal{U}}
\newcommand{\lU}{\Lambda_U}
\newcommand{\tP}{\tilde{\Pi}}
\newcommand{\rP}{\rho_{\Pi}}
\newcommand{\rtP}{\rho_{\tP}}


\newcommand{\bmL}{\boldsymbol{\mathcal{L}}}
\newcommand{\bmN}{\boldsymbol{\mathcal{N}}}
\newcommand{\pt}{\partial_t}
\newcommand{\mrp}{\mathrm{p}}

\newcommand{\Ad}{\mathrm{ad}}
\newcommand{\mud}{\mu^{\mathrm{d}}}
\newcommand{\mug}{\mu^{\mathrm{g}}}
\DeclareMathOperator{\erf}{Erf}
\DeclareMathOperator{\Lap}{Lap}
\DeclareMathOperator{\Sp}{Span}
\DeclareMathOperator{\tvec}{vec}

%Greek Letters
\def\a{\alpha}
\def\b{\beta}
\def\g{\gamma}
\def\d{\delta}
\def\e{\epsilon}
\def\ve{\varepsilon}
\def\z{\zeta}
\def\h{\eta}
\def\t{\theta}
%\def\i{\iota}
\def\k{\kappa}
%\def\l{\lambda}
\def\m{\mu}
\def\n{\nu}
\def\x{\xi}
\def\p{\pi}
\def\r{\rho}     %%%% IMPORTANT TEX NOTE: THE REDEFINITION FOR '\r' BREAKS THE ANGSTROM SYMBOL
\def\s{\sigma}
\def\ta{\tau}
\def\u{\upsilon}
\def\ph{\varphi}
\def\c{\chi}
\def\ps{\psi}
\def\o{\omega}

\def\G{\Gamma}
\def\D{\Delta}
\def\T{\Theta}
\def\L{\Lambda}
\def\X{\Xi}
\def\P{\Pi}
\def\S{\Sigma}
\def\U{\Upsilon}
\def\Ph{\Phi}
\def\Ps{\Psi}
\def\O{\Omega}

%Some helpful quantum shortcuts
\def\ox{\otimes}
\def\>{\rangle}
\def\<{\langle}
\def\Tr{\mathrm{Tr}}
\def\Pr{\mathrm{Pr}}
\newcommand{\ketbra}[1]{|{#1}\>\!\<#1|}
\newcommand{\bracket}[1]{\<{#1}|{#1}\>}
\newcommand{\bk}[2]{\<{#1}|{#2}\>}
\newcommand{\brak}[2]{\<{#1}|{#2}\>}
\newcommand{\ketb}[2]{|{#1}\>\!\<#2|}
\newcommand{\ketbsub}[3]{|{#1}\>_{#3}\<#2|}
\newcommand{\ident}{\mathbb{}}
\newcommand{\on}[1]{\|#1\|}

\newcommand{\opI}{\ensuremath{{{I}}}}
\newcommand{\opA}{\ensuremath{{{A}}}}
\newcommand{\opB}{\ensuremath{{{B}}}}
\newcommand{\opX}{\ensuremath{{{X}}}}
\newcommand{\opY}{\ensuremath{{{Y}}}}
\newcommand{\opZ}{\ensuremath{{{Z}}}}
\newcommand{\opS}{\ensuremath{{{S}}}}
\newcommand{\ee}{\ensuremath{{e}}} %e:=lim_{n\to\infty}(1+1/n)^n

\def\lp{\left(}
\def\rp{\right)}
\def\ls{\left[}
\def\rs{\right]}
\def\lb{\left\{}
\def\rb{\right\}}

\newcommand{\tcU}{\tilde{\mathcal{U}}}
\newcommand{\cV}{\mathcal{V}}
\newcommand{\tcV}{\tilde{\mathcal{V}}}
\newcommand{\rtcV}{\r_{\tcV}}
\newcommand{\rcV}{\r_\cV}
\newcommand{\sk}{\sigma_k}
\newcommand{\tsk}{\tilde{\sk}}
\newcommand{\cF}{\mathcal{F}}


\def\dgr{\dagger}

\def\Ito{It$\hat{\text{o}}$}

\DeclareMathOperator*{\MapWith}{\to}

\usepackage{tikz-cd}

\newcounter{postulate}[section]
\newenvironment{postulate}[1][]{\refstepcounter{postulate}\par\medskip
   \textbf{Postulate~\thepostulate. #1} \rmfamily}{\medskip}


%%%%%%%%%%%%%
